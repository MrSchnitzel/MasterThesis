\chapter{\abstractname}
% Looking closer at biological processes will often reveal interesting mechanisms. They most often evolved over time and helped the organism evoling them to survive by having a little edge on their contrahents. Trying to understand them and copying them already lead to great achiev
In dieser Arbeit werden die Möglichkeiten von Spiking Neural Networks für den Einsatz in der Automatisierung untersucht. Das Ziel ist es, einen Controller für einen schlangenförmigen Roboter mit gleitender Bewegung zu entwickeln. Um seine Umgebung wahrzunehmen ist der Roboter mit einem Dynamic Vision Sensor ausgestattet. Dieser Sensor hat sehr hohe Zeitaufloesung und benötigt relativ wenig Datenvolumen, da er Aktualisierungen nicht in festgelegten Abständen schickt, sondern nur bei signifikanten Änderungen einzelner Pixel. Die Ereignisse, die von der Kamera gesendet werden, können fast ohne Overhead in Impulse für das Netzwerk umgewandelt werden. Die Kombination dieser beiden Technologien ist vielversprechend für den Einsatz in autonomen Plattformen, da beide einen geringeren Energiebedarf als andere vergleichbare Lösungen haben. Außerdem arbeiten Spiking Neural Networks nicht in festen takten sondern zeitkontinuierlich was dem Roboter eine geringere Latenz in der ausfuehrung von Aktionen ermöglicht. Zudem ist der schlangenförmige Roboter ist sehr wendig und eignet sich hervorragend für den Einsatz in schwierigem Gelände.
In dieser Versuchsanordnung sind die Bewegungen des Schlangenkopfes eine Herausforderung, da die Eingaben des Sensors sich durch die Bewegung kontinuierlich verändern. Also war das erste Ziel dieser Arbeit, ein einfaches impulsgesteuertes neuronales Netz zu bauen, das den Roboter steuert und lernt einem object zu folgen. Als nächster Schritt soll das Steuerungsmodel des Roboters erweitert werden so, dass das Netzwerk den Kopf unabhängig vom Körper steuer kann. Netzwerke von verschidenartiger Topologie werden auf ihre Fähigkeit getestet diese komplexere Aufgabe zu beweltigen. Unter den getesteten Netzwerken befinden sich sowohl einlagige als auch mehrlagige Netzwerke.
Diese Ergbnisse dieser Arbeit zeigen, dass das vom Aufbau einfachste Netzwerk in der Lage war dem Ziel zu folgen; in komplexeren Steueraufgaben jedoch keines der einlagigen Netzwerke fähig war dem Ziel zu folgen. Dies geling nur einem mehrlagigen Netzwerk welches in dem komplizierteren Aufbau ähnliche Ergebnisse zielen konnte als das einlagige
bei der einfacheren Aufgabe.

%TODO: Abstract


