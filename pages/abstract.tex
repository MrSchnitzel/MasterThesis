\chapter{\abstractname}
% Looking closer at biological processes will often reveal interesting mechanisms. They most often evolved over time and helped the organism evoling them to survive by having a little edge on their contrahents. Trying to understand them and copying them already lead to great achiev


In this thesis the capabilities of Spiking Neural Networks for the use in robotics is studied. The aim is to realize a controler for a snake like robot with a slithering movement gait. To sense its environment the robot is equipted with a Dynamic Vision Sensor. This is an optical sensor which has very high temporal resultion and needs far less data volume to work as it doesn't send frames at a fixed rate but events if a significant change in a pixel ocurrs. The events send form the camera can be converted to input spikes for the network with almost no overhead. The combination of these technologies is promising for the use in autonomus platforms as both have a lower energie consumption then comparable solutions. Additionally spiking neural networks work continous in time which enables a smaller latency for the robot and it can react faster on a changing situation. The snake robot has very versatile movement capabilities which makes the concept interesting especially for difficult terrain.
\newline
In this setup movement of the snakes head while slithering is a special challange as the sensor input changes through out the movement. Thus the first aim in this thesis is to set up a simple spiking neural network to control the robot in a object following task. The next aim is to evolve the control model and enable the network to control the head independent from the movement direction. The capabilities of controling this more complex setup are tested for different single and multilayerd network topologies. The networks use reward modulated STDP synapses and reward backpropagtion is used for the multilayerd networks.
The results show that the simplest network is capable to precisely control the snake to follow a target. In the more complex control task on the other hand, no singlelayerd network was able to follow the target. This was only achieved by a multilayerd network which had a similar performance to the singlelayerd network in the more simple setup


%TODO: Abstract


