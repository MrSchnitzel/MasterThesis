\chapter{Conclusion}


% \begin{table}[htpb]
%   \caption[Parameters 2.Setup]{Parameters of body Neurons in the 2. setup} \label{tab:ParamsBase2N}
%   \begin{tabular}{|c| c |l|}
%       \toprule
%       Parameter & Value & Description \\
%       \midrule
%       $c_m$   & 20.0  & Capacity of the membrane \\
%       $tau_{m}$    & 50.0  & Membrane time constant \\
%       $tau_{refrac}$   & 1.  & Duration of refractory period\\
%       $v_{thresh}$   & -50.0  & Spike initiation threshold \\
%       $v_{reset}$    & -65.0  &  Reset value for $V_m$ after a spike \\
%       $v_{rest}$ & -65.0 & Resting voltage for $V_m$ \\
%       \bottomrule
%     \end{tabular}
%     \end{table}
%   \begin{table}[htpb]
%     \caption[Parameters 2.Setup]{Parameters of the body Synapses for the 2. setup} \label{tab:ParamsBase2S}
%     \begin{tabular}{|c| c |l|}
%         \toprule
%         Parameter  & Value & Description \\
%         \midrule
%         $W_{max}$ & 6000   & Maximum weight of synapse\\   
%         $W_{min}$ & -6000  & Minimum weight of synapse\\   
%         $A_{+}$   & 0.1    & Constant scaling strength of potentiation\\   
%         $A_{-}$   & -0.1   & Constant scaling strength of depression \\   
%         $\tau_c$  & 100.0   & Time constant of eligibility trace \\  
%         $\tau_n$  & 20.0   & Time constant of reward signal  \\   
%         $b$       & 0.0    & Baseline neuromodulator concentration \\    
%         \bottomrule
%     \end{tabular}
%     \end{table}


% \begin{figure}[htpb]
%   \centering
%   % This should probably go into a file in figures/
%   \begin{tikzpicture}[node distance=0.5cm]
%     \node (P0) {};
%     \node (P1) [right of=P0] {};
%     \node (P2) [right of=P1] {};
%     \node (P3) [right of=P2] {};
%     \node (P4) [right of=P3] {};
%     \node (P5) [right of=P4] {};
%     \node (P6) [right of=P5] {};
%     \node (P7) [right of=P6] {};
%     \node (P8) [right of=P7] {};
%     \node (P9) [right of=P8] {};

%     \node (I0) [below of=P0] {};
%     \node (I1) [right of=I0] {};
%     \node (I2) [right of=I1] {};
%     \node (I3) [right of=I2] {};
%     \node (I4) [right of=I3] {};
%     \node (I5) [right of=I4] {};
%     \node (I6) [right of=I5] {};
%     \node (I7) [right of=I6] {};
%     \node (I8) [right of=I7] {};
%     \node (I9) [right of=I8] {};

%     \node (O0) [below of=I0] {};
%     \node (O1) [right of=O0] {};

%     \node (D0) [below of=O0] {};
%     \node (D1) [below of=O1] {};

%     \path[every node]
%       (P0) edge (I0)
%       (P1) edge (I1)
%       (P2) edge (I2)
%       (P3) edge (I3)
%       (P4) edge (I4)
%       (P5) edge (I5)
%       (P6) edge (I6)
%       (P7) edge (I7)
%       (P8) edge (I8)
%       (P9) edge (I9)

%       (I0) edge (O0)
%       (I1) edge (O0)
%       (I2) edge (O0)
%       (I3) edge (O0)
%       (I4) edge (O0)
%       (I5) edge (O0)
%       (I6) edge (O0)
%       (I7) edge (O0)
%       (I8) edge (O0)
%       (I9) edge (O0)
      
%       (I0) edge (O1)
%       (I1) edge (O1)
%       (I2) edge (O1)
%       (I3) edge (O1)
%       (I4) edge (O1)
%       (I5) edge (O1)
%       (I6) edge (O1)
%       (I7) edge (O1)
%       (I8) edge (O1)
%       (I9) edge (O1)

%       (O0) edge (D0)
%       (O1) edge (D1);
%   \end{tikzpicture}
%   \caption[Simple Network]{Simple Network Topology}\label{fig:simpleNetwork}
% \end{figure}
